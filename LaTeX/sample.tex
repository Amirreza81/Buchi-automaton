\documentclass[17pt, t, lualatex]{beamer}

\title{$\omega$-Automata, B$\ddot{u}$chi and Generalized B$\ddot{u}$chi Automata}
\date{Spring 2024}
\institute[SUT]{Sharif University of Technology}
\author{AmirReza Azari}

\usepackage{amsmath,amssymb,mathtools}

% Probably load as late as possible
% Other options are
% - engine=pdflatex to compile in pdfLaTeX (with different fonts),
% - mathshape=rm to use serif font for math,
% - mathsahpe=custom to not set any math font (so that you can define your own math fonts)
\usetheme[engine=lualatex, mathshape=sf, fontdir=kthpq-files/fonts/Figtree/]{kthpq}
\setmonofont{Bitstream Vera Sans Mono}[Scale=.9]

% Custom colors (see beamercolorthemecustom.sty for more details)
%\usecolortheme{custom}

% Custom footline
%\setfootline{left}{center}{right}

\begin{document}

\inserttitlepage

\section{$Omega\;\; Languages$}

\insertsectionpage

\begin{frame}{$\omega$-Automata}
\begin{block}{Definition:}
\begin{itemize}
    \item
    Automata that accept (or reject) words of 
infinite length.
    \item 
    Languages of infinite words appear:
    \begin{itemize}
        \item
        In verification, as encodings of non-terminating 
executions of a program.
        \item 
        In arithmetic, as encodings of sets of real 
numbers.
    \end{itemize}
\end{itemize}
\end{block}
\end{frame}

\begin{frame}{$\omega$-Regular Languages}
\begin{block}{Definition:}
\begin{itemize}
    \item Infinite words over the alphabet $\Sigma$ are infinite sequences $A_0, A_1, A_2, \ldots$ of symbols $A_i \in \Sigma$.
    \item $\Sigma^{\omega}$ denotes the set of all infinite words over $\Sigma$.
    \item Any subset of $\Sigma^{\omega}$ is called a language of infinite words, called an $\omega$-language.
    \item For instance, the infinite repetition of the finite word AB yields
    the infinite word $ABABABABAB\ldots$ (ad infinitum) and is denoted by $(AB)^{\omega}$.
    \item For the special case of the empty word, we have $\varepsilon^{\omega} = \varepsilon$. 
    \item 
    For an infinite word, infinite repetition has no effect, that is $\sigma^{\omega} = \sigma$ if $\sigma \in \Sigma^{\omega}$.

\end{itemize}
\end{block}
\end{frame}

\begin{frame}{$\omega$-Regular Expression}
\begin{block}{Definition:}
\begin{itemize}
    \item
    An $\omega$-regular expression G over the alphabet $\Sigma$ has the form:
    \begin{center}
        $G = E_1.F_1^{\omega} + \ldots + E_n.F_n^{\omega}$
    \end{center}
    where $n \geq 1$ and $E_1,\ldots,E_n,F_1,\ldots,F_n$ are regular expressions over $\Sigma$ such that $\varepsilon \notin \pazocal{L}(F_i)$, for all $1 \leq i \leq n$.
    \item 
    If $\pazocal{L}(E) \subseteq \Sigma^*$ denotes the language (of finite words) induced by the regular expression E: 
    \begin{center}
        $\pazocal{L}_{\omega}(G) = \pazocal{L}(E_1).\pazocal{L}(F_1)^{\omega} \smallcup \ldots \smallcup \pazocal{L}(E_n).\pazocal{L}(F_n)^{\omega}$
    \end{center}
\end{itemize}
\end{block}
\end{frame}

\begin{frame}{$\omega$-Regular Expression}
\begin{block}{Definition (cont.)}
\begin{itemize}
    \item 
    Example for $\omega$-regular expressions over the alphabet 
    $\Sigma = \{A,B,C\}:$\\[0.4in]
    \begin{center}
        $(A+B)^*A(AAB+C)^{\omega}$ \\[0.15in]
        or\\[0.15in]
        $A(B+C)^*A^{\omega} + B(A+C)^{\omega}$
    \end{center}
\end{itemize}
\end{block}
\end{frame}

\begin{frame}{$\omega$-Regular Expression}
\begin{block}{Example:}
\begin{itemize}
    \item[1.]
    A word in $aa\Sigma^*aa$ followed by only $b$ $\rightarrow$ 
    $aa\Sigma^*aa.b^{\omega}$
    \begin{center}
        $\{aaaabbbb\ldots\},\; \{aabbbbaaabbbb\ldots\}$
    \end{center}
    \item[2.]
    Infinite words where b occurs only finitely often $\rightarrow$ $(a+b)^*.b^{\omega}$
    \begin{center}
        $\{aaaa\ldots\},\; \{babbaaaa\ldots\}$
    \end{center}
    \item[3.]
    Infinite words where b occurs infinitely often $\rightarrow$
    $(a^*b)^{\omega}$
    \begin{center}
        $\{abababab\ldots\},\;\{bbbabbbabbbabbba\ldots\},\;
        \{bbbbbbbbbbbbb\ldots\}$
    \end{center}
\end{itemize}
\end{block}
\end{frame}

\begin{frame}{$\omega$-Regular Expression}
\begin{block}{More Examples:}
\begin{itemize}
    \item[1.]
     $(a + b)^{\omega}$ set of all infinite words.
     \item[2.]
     $a(a + b)^{\omega}$ infinite words starting with an a
     \item[3.]
      $(a + bc + c)^{\omega}$ words where every b is immediately followed by c
      \item[4.]
      $((a + b)^*c)^{\omega}$ words where c occurs infinitely often
      \item[5.]
      $(a + b)^*c(a + b)^{\omega}$ words with a single occurrence of c
\end{itemize}
\end{block}
\end{frame}

\begin{frame}{Regular Languages}
    \begin{figure}
        \centering
        \includegraphics[scale=0.6]{1.png}
    \end{figure}
\end{frame}

\begin{frame}{$\omega$-Regular Languages}
    \begin{figure}
        \centering
        \includegraphics[scale=0.6]{2.png}
    \end{figure}
\end{frame}

\section{$B\ddot{u}chi\;\; Automata$}

\insertsectionpage

\begin{frame}{$Run\;and\;acceptance$}
\begin{figure}
    \centering
    \includegraphics[scale=0.85]{4.png}
\end{figure}
\end{frame}

\begin{frame}{$Run\;and\;acceptance$}
\begin{block}{Run is accepting if some accepting state occurs infinitely often}
Below word is not accepted by this automaton.
\begin{figure}
    \centering
    \includegraphics[scale=0.85]{5.png}
\end{figure}
\end{block}
\end{frame}

\begin{frame}{$Non-deterministic\; B\ddot{u}chi\; Automata$}
\begin{block}{Definition. Non-deterministic B$\ddot{u}$chi Automata (NBA)}
A Non-deterministic B$\ddot{u}$chi automaton $\pazocal{A}$ is a tuple $\pazocal{A} = (Q, \Sigma, \delta, Q_0, F)$ where:
\begin{itemize}
    \item Q is a finite states,
    \item $\Sigma$ is an alphabet,
    \item $\delta$: $Q \times \Sigma \rightarrow 2^Q$ is a transition function,
    \item $Q_0 \subseteq Q$ is a set of initial states 
    \item $F \subseteq Q$ is a set of accept states, called acceptance set.
    
\end{itemize}
A language $L \subseteq \Sigma^{\omega}$ is $\omega$-regular if it can be accepted by some B$\ddot{u}$chi automaton.
\end{block}
\end{frame}

\begin{frame}{$Non-deterministic\; B\ddot{u}chi\; Automata$}
\begin{block}{Definition (cont.)}
\begin{figure}
    \centering
    \includegraphics[scale=1.1]{6.png}
\end{figure}
\end{block}
\end{frame}

\begin{frame}{NFA vs. NBA}
\begin{itemize}
    \item Syntax differences between NFA and NBA : None\\
    
    \item Semantics differences between NFA and NBA: the accepted 
language of an NFA A is a language of finite words, whereas the 
accepted language of NBA A is an $\omega$-language. \\

The intuitive meaning of the acceptance criterion named after Buchi is that the accept set of A has to be visited infinitely often. Thus, the accepted language L$\omega$(A) consists of all infinite words that have a run in which some accept state is visited infinitely often.
\end{itemize}
\end{frame}

\begin{frame}{NBA}
\begin{block}{Example}
    $a^{\omega} + b^{\omega}:$
    \begin{figure}
        \centering
        \includegraphics{7.png}
    \end{figure}
\end{block}
\end{frame}

\begin{frame}{NBA}
\begin{block}{Example}
    $aa(a+b)^*ab^{\omega}:$
    \begin{figure}
        \centering
        \includegraphics{8.png}
    \end{figure}
\end{block}
\end{frame}

\begin{frame}{NBA - $\omega$-operator}
    \begin{figure}
        \centering
        \includegraphics[scale=0.8]{9.png}
    \end{figure}
\end{frame}

\begin{frame}{NBA - Concatenation}
A language $L_1$ can be concatenated with an $\omega$-language $L_2$ to 
yield the $\omega$-language $L_1L_2$, but two $\omega$-languages cannot be concatenated.
    \begin{figure}
        \centering
        \includegraphics[scale=0.5]{10.png}
    \end{figure}
\end{frame}

\begin{frame}{NBA - Union}
    \begin{figure}
        \centering
        \includegraphics[scale=0.65]{11.png}
    \end{figure}
\end{frame}

\begin{frame}{NBA to $\omega$-regular expression}
\begin{block}{Lemma.}
 Let $A$ be a NFA, and let $q,\;q'$
be states of $A$.\\
The language $L_q^{q'}$ of words with runs leading from $q$ to $q'$ and visiting $q'$ exactly once after leaving $q$ is regular.
\begin{itemize}
    \item Let $r_q^{q'}$ denote a regular expression for $L_q^{q'}$.
    \item Given a NBA $A$, we look at it as a NFA, and compute regular expressions $r_q^{q'}$.
    \item We show:
    \begin{center}
        $L_{\omega}(A)=L\left(\sum_{q\in F} r_{q_0}^q (r_q^q)^{\omega}\right)$
    \end{center}
\end{itemize}
\end{block}    
\end{frame}

\begin{frame}{NBA to $\omega$-regular expression}
\begin{block}{Example:}
    \begin{figure}
        \centering
        \includegraphics[scale=0.5]{12.png}
    \end{figure}
\end{block}
\end{frame}

\begin{frame}{Constructing a NBA from a NFA}
    \begin{figure}
        \centering
        \includegraphics{13.png}
        \includegraphics{14.png}
    \end{figure}
\end{frame}

\begin{frame}{Constructing a NBA from a NFA}
    \begin{figure}
        \centering
        \includegraphics[scale=1]{15.png}
    \end{figure}
\end{frame}

\begin{frame}{Constructing a NBA from a NFA}
    \begin{figure}
        \centering
        \includegraphics[scale=1]{16.png}
    \end{figure}
\end{frame}

\begin{frame}{$Deterministic\; B\ddot{u}chi\; Automata$}
\begin{block}{Definition:}
    \begin{itemize}
        \item Single initial state
        \item From every state - on an alphabet, there is a unique transition
    \end{itemize}
    \begin{figure}
        \centering
        \includegraphics[scale=0.7]{17.png}
    \end{figure}
\end{block}
\end{frame}

\begin{frame}{DBA}
\begin{block}{Question: Can every NBA be converted to an equivalent
DBA?}
\begin{figure}
    \centering
    \includegraphics{18}
\end{figure}
\end{block}
\end{frame}

\begin{frame}{DBA}
\begin{block}{DBA less powerful than NBA}
    \begin{itemize}
        \item Automaton has to guess the point from where only b occurs
        \item A deterministic B$\ddot{u}$chi automaton cannot make this guess
        \item The above language cannot be accepted by a DBA.
    \end{itemize}
    \begin{block}{Proof.}
         By contradiction. Assume some DBA recognizes $(a+b)^*b^{\omega}$.
    \end{block}
\end{block}   
\end{frame}

\begin{frame}{DBA}
\begin{block}{Proof (cont.)}
    \begin{figure}
        \centering
        \includegraphics[scale=0.7]{19.png}
    \end{figure}
\end{block}
\end{frame}

\begin{frame}{$\omega$-Regular languages}
\begin{figure}
    \centering
    \includegraphics[scale=0.7]{3.png}
\end{figure}
\end{frame}


\section{$Generalized\; B\ddot{u}chi\; Automata$}

\insertsectionpage

\begin{frame}{GBA}
\begin{block}{Definition.}
\begin{itemize}
    \item Generalized B$\ddot{u}$chi automaton (GBA) is a variant of B$\ddot{u}$chi automaton
    \item The difference with the B$\ddot{u}$chi automaton is its accepting
    condition, i.e., a set of sets of states.
    \item A run is accepted by the automaton if it visits at least one state of
    every set of the accepting condition infinitely often.
    \item  Generalized B$\ddot{u}$chi automata (GBA) is equivalent in expressive power with B$\ddot{u}$chi automata

\end{itemize}
\end{block}
\end{frame}

\begin{frame}{GBA}
\begin{block}{Definition (cont.)}
\begin{itemize}
    \item 
    A generalized Buchi automaton (GBA) over $\Sigma$ is:
    \begin{center}
        $A = (Q, \Sigma, \delta, I, F)$
    \end{center}
    \item Q is a finite set of states
    \item $\Sigma = \{a, b, \ldots\}$ is a finite alphabet set of A
    \item $\delta \subseteq Q \times \Sigma \times Q$ is a transition relation
    \item $I \subseteq Q$ is a set of initial states
    \item $F = \{F_1,\ldots,F_k\} \subseteq 2^Q$ is a set of sets of final states.
     
\end{itemize}
\end{block}
\end{frame}

\begin{frame}{GBA}
\begin{block}{Definition (cont.)}
    \begin{itemize}
        \item A accepts exactly those runs in which the set of infinitely often occurring states contains at least a state from each $F_1,\ldots,F_n$.
        \item A run $\sigma$ of a GBA is said to e accepting iff,
        \begin{center}
            for all $1\leq i \leq k$, we have inf($\sigma$) $\cap\;$ F$_i$ $\neq \emptyset$.
        \end{center}
    \end{itemize}
    \begin{block}{inf($\sigma$)}
        The set of states visited infinitely often by a run $\sigma$ is denoted inf($\sigma$).
    \end{block}
\end{block}
\end{frame}

\begin{frame}{GBA}
\begin{block}{(inf($\sigma$) (cont.))}
    \begin{figure}
        \centering
        \includegraphics[scale=0.8]{20.png}
    \end{figure}
\end{block}
\end{frame}

\begin{frame}{NBA $\&$ GNBA}
    \begin{figure}
        \centering
        \includegraphics[scale=0.8]{21.png}
        \includegraphics[scale=0.8]{22.png}
    \end{figure}
\end{frame}

\begin{frame}{NBA $\&$ GNBA}
    A GNBA for the property ”both processes are infinitely often in their critical section”:
    \begin{figure}
        \centering
        \includegraphics[scale=1]{23.png}
    \end{figure}
    \begin{itemize}
        \item GNBA are like NBA, but have a distinct acceptance criterio. A GNBA requires to visit several sets $F_1, \ldots, F_k (k \geq 0)$ infinitely often.
    \end{itemize}
\end{frame}

\begin{frame}{De-generalization of GBA}
    \begin{figure}
        \centering
        \includegraphics[scale=1]{24.png}
        \includegraphics[scale=0.95]{25.png}
    \end{figure}
\end{frame}

\begin{frame}{De-generalization of GBA}
\begin{block}{Algorithm:}
    \begin{itemize}
        \item Turn a generalized B$\ddot{u}$chi automaton into a B$\ddot{u}$chi automaton.
        \item The idea:
        \begin{itemize}
            \item Each cycle must go through every copy.
            \item Each cycle must contain accepting states from each accepting set.
        \end{itemize}
        \item Algorithm:
        \begin{itemize}
            \item Duplicate the GBA to as many copies as the number of accepting sets
            \item Redirect outgoing edges from accepting states to the next copy.
        \end{itemize}
    \end{itemize}
\end{block}
\end{frame}

\begin{frame}{De-generalization of GBA}
\begin{block}{Example:}
    \begin{figure}[ht]
        \subcaptionbox*{1,2 correspond to $F_1$ and $F_2$, the accepting sets}[.45\linewidth]{%
        \includegraphics[scale=0.5]{26.png}%
    }%
    \hfill
        \subcaptionbox*{Two copies, because we have two accepting sets}[.45\linewidth]{%
        \includegraphics[scale=0.5]{27.png}%
    }
\end{figure}
\end{block}
\end{frame}

\begin{frame}{De-generalization of GBA}
\begin{block}{Example (cont.):}
    \begin{figure}[ht]
        \subcaptionbox*{Choose one cope as initial and redirect edges from accepting edges}[.45\linewidth]{%
        \includegraphics[scale=0.5]{28.png}%
    }%
    \hfill
        \subcaptionbox*{remove unreachable states}[.45\linewidth]{%
        \includegraphics[scale=0.5]{29.png}%
    }
\end{figure}
\end{block}
\end{frame}

\begin{frame}{De-generalization of GBA}
\begin{block}{Another Example:}
    \begin{figure}[ht]
        \subcaptionbox*{A GBA}[.45\linewidth]{%
        \includegraphics[scale=0.7]{30.png}%
    }%
    \hfill
        \subcaptionbox*{One copy for each accepting set}[.45\linewidth]{%
        \includegraphics[scale=0.6]{31.png}%
    }
\end{figure}
\end{block}
\end{frame}

\begin{frame}{De-generalization of GBA}
\begin{block}{Another Example (cont.):}
    \begin{figure}[ht]
        \subcaptionbox*{Redirecting edges}[.45\linewidth]{%
        \includegraphics[scale=0.6]{32.png}%
    }%
    \hfill
        \subcaptionbox*{and so forth...}[.45\linewidth]{%
        \includegraphics[scale=0.6]{33.png}%
    }
\end{figure}
\end{block}
\end{frame}

\begin{frame}{De-generalization of GBA}
\begin{block}{Another Example (cont.):}
    \begin{figure}
        \centering
        \includegraphics[scale=1]{34.png}
    \end{figure}
\end{block}
\end{frame}

\section{$Review$}

\insertsectionpage

\begin{frame}{$\omega$-automata}
    \begin{figure}[ht]
        \subcaptionbox*{}[.45\linewidth]{%
        \includegraphics[scale=1.1]{r1.png}%
    }%
    \hfill
        \subcaptionbox*{}[.45\linewidth]{%
        \includegraphics[scale=1]{r2.png}%
    }
\end{figure}
\end{frame}

\begin{frame}{B$\ddot{u}$chi Automata}
    \begin{figure}
        \centering
        \includegraphics[scale=1.5]{r3.png}
    \end{figure}
\end{frame}

\begin{frame}{B$\ddot{u}$chi Automata}
    \begin{figure}
        \centering
        \includegraphics[scale=1.5]{r4.png}
    \end{figure}
\end{frame}

\begin{frame}{Generalized B$\ddot{u}$chi Automata}
    \begin{figure}
        \centering
        \includegraphics[scale=1.5]{r5.png}
    \end{figure}
\end{frame}

\begin{frame}{Generalized B$\ddot{u}$chi Automata}
    \begin{figure}
        \centering
        \includegraphics[scale=1.5]{r6.png}
    \end{figure}
\end{frame}

\section{$References$}

\insertsectionpage

\begin{frame}{}
    
\begin{itemize}
    \item https://www.cmi.ac.in/~kumar/words/lecture07.pdf
    \item https://www.cs.colostate.edu/~france/CS614/Slides/ModelCheckingChapter4.pdf
    \item https://www.youtube.com/watch?v=KOu6IUssxbs
    \item https://www.lrde.epita.fr/~sadegh/buchi-complementation-techrep.pdf
    \item https://www.irif.fr/~jep/PDF/InfiniteWords/Chapter1.pdf
\end{itemize}

\end{frame}

\insertendpage

\end{document}